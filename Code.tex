\documentclass{article}

\usepackage[usenames]{color}
\usepackage[russian]{babel}
\usepackage[T2A]{fontenc}
\usepackage[utf8]{inputenc}
\usepackage{authblk}
\usepackage[dvipsnames]{xcolor}
\usepackage[T1]{fontenc}

\title{Code Style}
\author{ИА-031 Обухович Ярослав, email: yar.obuxowi42002@gmail.com,  github: @Digital1804}
\date{Февраль 2022}

\begin{document}

\maketitle

\section{Введение}
Стиль кода (code style) — это набор правил, как писать код в проекте. Они могут включать прямые рекомендации, примеры кода, ссылки на лучшие практики, «рецепты», что делать в спорных ситуациях. \cite{one}

\section{C/C++}
\subsection{Отступы в начале строки}

Для отступов используется табуляция. Размер таба равен 4 пробелам.\\\\
\newcommand*{\Tabulate}{\hspace*{0.5cm}}
\begin{lstlisting}
\begin{tabular}{ | l | }
\hline
\\
int main() \{\\
\Tabulate return 0;\\
\}\\
\\
\hline
\end{tabular}
\end{lstlisting}
\subsection{Пробелы}
Используется 1 пробел между ключевым словом и открывающей скобкой:\\\\
\begin{lstlisting}
\begin{tabular}{ | l | }
\hline
\\
if (condition)...\\
while (condition)...\\
for (initialisation; condition; step)...\\
do \{\} while (condition)...\\
\\
\hline
\end{tabular}\\
\\\\
Пробел всегда используется после знаков препинания.\\\\
Используется один пробел до и после операторов сравнения и присваивания:\\
До и после операторов сравнения используется отступ в один пробел:\\\\
\begin{lstlisting}
\begin{tabular}{ | l | }
\hline
\\
int number = 0;\\
for (int i = 0; i < size; i++) \{\}\\
\\
\hline
\end{tabular}
\end{lstlisting}
\\
\subsection{Объявление имен переменных, функций, структур и классов}
Функции назваются в соответствии с целью ее работы, переменные, структуры и классы - с целью использования, слитно, заменяя пробелы на нижнее подчеркивание\\\\
\begin{lstlisting}
\begin{tabular}{ | l | }
\hline
\\int list[3] = [1, 2, 3];
\\struct Node* node = NULL;
\\void add\_to\_list(...);
\\int sort\_tree(...);
\\\\
\hline
\end{tabular}
\end{lstlisting}
\newpage
\subsection{Функции}
В порядке написании функций стоит придерживаться последовательности применения, чтобы не было ошибок.\\\\
Между функциями обязательно оставляется пустая строка для более удобного листинга программы.\\\\
Закрывающая скобка всегда ставится в соответствии с уровнем начала функции, к которой она принадлежит\\\\
\begin{lstlisting}
\begin{tabular}{ | l | }

\hline
\\
void \colorbox{Cyan}{division}(double **matrix, int size) \{\\
    \Tabulate ...
\}\\\\

void \colorbox{Yellow}{print\_matrix}(double **matrix, int size) \{\\
    \Tabulate ...
\}\\\\

void gauss(Node *root, string *buffer) \{\\
	\Tabulate ...\\
	\Tabulate \colorbox{Cyan}{division}(matrix, size);\\
	\Tabulate ...\\
	\Tabulate \colorbox{Yellow}{print\_matrix}(matrix, size);\\

\}\\
\\
\hline

\end{tabular}
\\\\
\end{lstlisting}
\newpage
\subsection{Циклы}\\
Для инициализации счетчика циклов используются переменные i, j, k. между оператором и параметрами цикла всегда ставится пробел.\cite{two}\\\\
\large \textbf{for}\\\\
\normalsize
\begin{lstlisting}
\begin{tabular}{ | l | }

\hline
\\
for (int i = 0; i < SIZE; i++) \{\\
         \Tabulate for (int j = 0; j < SIZE; j++) \{\\
               \Tabulate\Tabulate.....................; \\
           \Tabulate \} \\
\}
\\\\
\hline

\end{tabular}
\\\\
\end{lstlisting}
\\\large \textbf{while}\\\\
\normalsize
\begin{left}
\begin{lstlisting}
\begin{tabular}{ | l | }

\hline
\\
while (i < SIZE) \{\\
               \Tabulate.....................; \\
\}
\\\\
\hline

\end{tabular}
\\\\
\end{lstlisting}\\
\end{left}\\\\
\\\large \textbf{do while}\\\\
\begin{left}
\begin{lstlisting}
\begin{tabular}{ | l | }

\hline
\\
do \{\\
               \Tabulate.....................; \\
        \} while (i < SIZE);
\\\\
\hline

\end{tabular}
\\\\
\end{lstlisting}
\end{left}\
\newpage
\subsection{switch}
Перед каждым вариантом ставится отступ. После ключевого слова осуществляется переход на новую строку и, начиная с этой строки ставится двойной отступ.\\\\
\begin{lstlisting}
\begin{tabular}{ | l | }

\hline
\\
switch (number) \{\\
    \Tabulate case 1:\\
        \Tabulate\Tabulate cout << \verb|"|first\verb|"| << endl;\\ 
        \Tabulate\Tabulate break;\\
    \Tabulate case 2:\\
        \Tabulate\Tabulate cout << \verb|"|second\verb|"| << endl;\\ 
        \Tabulate\Tabulate break;\\
    \Tabulate case 3:\\
        \Tabulate\Tabulate cout << \verb|"|third\verb|"| << endl;\\ 
        \Tabulate\Tabulate break;\\
\}
\\\\
\hline

\end{tabular}
\\
\end{lstlisting}
\subsection{if else}
Каждый if и else пишется на отдельной строчки(кроме случая else if, тогда они пишутся через пробел). Фигурные скобки ставятся только в случае, если при попадании в условие выполняется более одного действия, иначе оно пишется через отступ.\\\\
\begin{lstlisting}
\begin{tabular}{ | l | }

\hline
\\
if (a == b == c) \{\\
    \Tabulate cout << \verb|"|Числа равны\verb|"| << endl;\\
    \Tabulate return a;\\ 
\}\\
if (a >= b) \Tabulate return a;\\
else if (b >= c) \Tabulate return b;\\
else return c;\\
\\
\hline

\end{tabular}
\\\\
\end{lstlisting}
\subsection{Структуры}
Для удобства в с ипользования структур, он объявляются с помощью typedef \cite{three}. Сами названия пишутся с большой буквы.\\\\
\begin{lstlisting}
\begin{tabular}{ | l | }

\hline
\\typedef struct Node \{\\
    \Tabulate string *line;\\
    \Tabulate int height;\\
    \Tabulate struct Node *left;\\
    \Tabulate struct Node *right;\\
\} Node;
\\\\
\hline

\end{tabular}
\\\\
\end{lstlisting}
\subsection{Классы}
\begin{lstlisting}
\begin{tabular}{ | l |}

\hline
\\

class Mobile \{\\
private:\\
    \Tabulate string MN;\\
    \Tabulate string type;\\
    \Tabulate int volume;\\
public:\\
    \Tabulate Mobile(string F, string t, int vol) \{\\
        \Tabulate\Tabulate MN = F;\\
        \Tabulate\Tabulate type = t;\\
        \Tabulate\Tabulate volume = vol;\\
    \Tabulate \}\\
    \Tabulate void output\_to() \{\\
        \Tabulate\Tabulate cout << \verb|"|Название мобильного устройства \verb|"| << MN << endl;\\
    \Tabulate \}\\
    \Tabulate void set\_MN(string f) \{\\
        \Tabulate\Tabulate MN = f;\\
    \Tabulate \}\\
    \Tabulate void set\_type(int t) \{\\
        \Tabulate\Tabulate type = t;\\
    \Tabulate \}\\
    \Tabulate void set\_volume(int vol) \{\\
        \Tabulate\Tabulate volume = vol;\\
    \Tabulate \}\\
\};\\
\\
\hline

\end{tabular}
\\\\
\end{lstlisting}
\newpage
\begin{thebibliography}{1}
\bibitem{one}Стиль написания кода в команде.\\ [https://doka.guide/js/code-style/]\\
\bibitem{two}Циклы. Операторы цикла\\
[https://www.bestprog.net/ru/2017/09/04/cycles-operators-of-the-cycle-for-while-do-while\_ru/]
\bibitem{three}Объявления Typedef
[https://docs.microsoft.com/ru-ru/cpp/c-language/typedef-declarations?view=msvc-16]0\\
\end{thebibliography}
\bibliography{sample}

\end{document}